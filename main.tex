\documentclass[pdftex, a4paper]{scrartcl}
%\usepackage{ngerman}
\usepackage[utf8]{inputenc}
\usepackage[T1]{fontenc}
%\usepackage{array}
%\usepackage{subfiles}
\usepackage{url}
\usepackage[hidelinks]{hyperref}
\usepackage{parskip}
\setlength{\parskip}{1em}
\usepackage{xurl}
\usepackage[nottoc,notlot,notlof]{tocbibind}
\usepackage{setspace}
\linespread{1.5}
\usepackage[
top    = 2.75cm,
bottom = 2.50cm,
left   = 3.00cm,
right  = 2.50cm]{geometry}
\usepackage{etoolbox}
\AtBeginEnvironment{thebibliography}{\linespread{1}\selectfont}
\usepackage{endnotes}
\interfootnotelinepenalty=10000
\usepackage[acronym,toc,nonumberlist]{glossaries} 
\usepackage[titletoc]{appendix}
\usepackage{booktabs}
\usepackage{tabularx}
\usepackage{graphicx}
%\usepackage{textcomp}
\usepackage{lscape}
\usepackage{fancyhdr} 
\usepackage{multirow}
\usepackage{caption}
\usepackage{natbib}
\usepackage[table,xcdraw]{xcolor}
\usepackage{float}
%\usepackage{lineno}
\usepackage{longtable}
\makeglossaries

\newacronym{fao}{FAO}{Food and Agriculture Organization of the United Nations}
\newacronym{un}{UN}{United Nations}
\newacronym{api}{API}{Application Programming Interface}


\newglossaryentry{user} {
    name={User},
    plural={users},
    description={See stakeholder}
}

\newglossaryentry{app} {
    name={App},
    description={It refers to the mobile application to be developed}
}
\newglossaryentry{stakeholder} {
    name={Stakeholder},
    plural={stakeholders},
    description={Describes all kind of potential person or entity that may have interest using the app}
}
\newglossaryentry{client} {
    name={Client},
    plural={clients},
    description={Since we have two major stakeholders that will use the app, the word client will
    specify the one that places an order in the app}
}
\newglossaryentry{provider} {
    name={Provider},
    plural={providers},
    description={The second major stakeholders are those who offer their products. They can be
                restaurants, bakeries, pastries and similar}
}
\newglossaryentry{upload} {
    name={Upload},
    description={Will designate the act of pushing textual or visual content into the app. It will
                be mostly done by a provider}
}
\newglossaryentry{register} {
    name={Register},
    description={Describes the act of inputting name, e-mail, company name, product or any other kind of
    textual data used to identify a stakeholder or a product}
}
\newglossaryentry{activity diagram} {
    name={Activity Diagram},
    description={This kind of diagram shows the behavior of a system, it depicts in a graphical fashion 
    the logic of a single use case \cite{refinbook:Baresi2009}.}
}
\newglossaryentry{class diagram} {
    name={Class Diagram},
    description={This kind of diagram presents the structure of a system with its classes, attributes,
    methods and relationships\cite{refonline:IBMCD}}
}
\newglossaryentry{use case diagram} {
    name={Use Case Diagram},
    description={This kind of diagram presents the main requirements and functionalities of a systems. It displays
    a simplified overview of core purpose of the application \cite{refart:YWRUS}}
}
\newglossaryentry{mobile payment gateway} {
    name={Mobile Payment Gateway},
    description={Those services works as an intermediary between customer, merchant and bank/credit card company. 
    Here a payment request is sent to the gateway and forwarded to the approval instances. The core functionality 
    of those gateways is the cryptography within the communication steps\cite{refonline:VPGI}}
}

\newglossaryentry{federated login} {
    name={Federated Login},
    description={Authentication method in which users use existing accounts to gain access to another domains or systems
    without the need of creating new credentials. The authenticity of a user is attested by service and granted to another 
    \cite{refonline:MRFL}}
}
\newglossaryentry{system response} {
    name={System Response},
    description={The output of a system after an input \cite{refonline:HWHE}.
    \cite{refonline:MRFL}}
}
\newglossaryentry{API} {
    name={Application Programming Interface - API},
    plural={APIs},
    description={Software intermediary that promotes the communication between different systems/applications/
    softwares\cite{refonline:MSAPI}}
}

% &as_qdr=y2 (find date)




\begin{document}
    \begin{titlepage}
    \vspace*{2mm}
    \begin{center}
        \Large
        \textbf{University for Applied Sciences}\\
        \textbf{Informatics Department}\\
        \textbf{Applied Informatics}\\
        \vspace{2cm}
        \textbf{To be defined}\\
        \vspace{2cm}
        \large
        Documentation for the Architecture of an Mobile Application for Preventing Food Waste\\
        \vspace{4cm}
        \begin {table}[ht]
            \centering
            \begin{tabular}{c}
                Bruno Macedo da Silva    \\ 
                676839                   \\
                inf3645@hs-worms.de      \\
            \end{tabular}
        \end {table}
        \vspace{2cm}
        \large
        \vspace{2cm}
         \begin {table}[ht]
             \centering
             \begin{tabular}{l l}
                Supervisor         & Prof. Dr. Volker Schwarzer \\
                Working Period:    & Summer Semester 2022 \\
                Due Date:          & 31.Juni 2022 \\
             \end{tabular}
         \end {table}
    \end{center}
    \normalsize
    \vfill
 


\end{titlepage}
    \tableofcontents
    %\newpage
    %\addcontentsline{toc}{section}{\listfigurename}
    %\listoffigures
    \newpage
    \clearpage
    \printglossary[type=acronym,title=Abbreviations,toctitle=Abbreviations]
    \clearpage
    \newpage
    \clearpage
    \printglossary[title=Glossary,toctitle=Glossary]
    \clearpage
    \section{Introduction and Goals}

According to the \acrfull{fao} in 2019 931 millions tonnes of food were wasted \cite{refart:FAOFW}. This has
environmental, but special social consequences. In a world were approximately 9.9\% of the \cite{refart:AAHWH}
population suffers from hunger that waste percentage sounds paradoxal.

According to \acrfull{un} 5\% of the globally food loss and waste comes from restaurants \cite{refart:UNSP}. 
The solution for this problem muss be locally applied so its effects can be seen in a global structure. To do so we
propose to develop a mobile application that connects restaurants, bakeries and or pastries to clients. 
The former would offer their remaining products, which are still consumable, prior to the closing time, to a small price 
and the latter would browser in the app to find which shops are offering products. 

 
\subsection{Design Purpose}

The main purpose of this architecture is creating exploratory prototype of an \gls{app}. We aim to test it with potential 
\gls{stakeholder} and regions to analyze the general their acceptance and wishes \cite{refbook:DSHC} and get a fast feedback. 

This prototype will also make it feasible to identify unknown needs an wishes of the potential \gls{stakeholder}, so we can eventually
increase the scope of functionality. Exploring this domain will also provide us with information regarding the behavior 
of our \gls{stakeholder} when it comes to buying and serving food that would be wasted, but is still consumable.

\subsection{Primary Functionality} \label{Primary_Functionality}

From the following use cases we will be able to define the primary functionality of our application and furthermore 
identify its main quality attributes 

% Object: put figure beside the table

\begin{table}[H]
    \begin{tabularx}{\textwidth}{lX}
    \toprule
    Use Case & Description  \\
    \midrule
    UC-1: Register as \gls{client} & The \gls{client} register an e-mail address.\\
    UC-2: Login & The \gls{client} logins in to the system. \\
    UC-3: Place an order & The \gls{client} chooses a \gls{provider}. \\
    UC-4: Register payment & The \gls{client} register a payment method. \\
    UC-5: Register as \gls{provider} & The \gls{provider} register their facility and products. \\
    UC-6: Update availability & The \gls{provider} upload their availability to provide a product. \\
    \bottomrule
    \end{tabularx}
\end{table}

Those use cases are also represented in the following use case diagram:

\begin{figure}[H]
    \centering
    \includegraphics[width=0.9\textwidth]{assets/preliminary_use_case.png}
    \caption{Preliminary functions}
    \label{fig:preliminary_use_case}
\end{figure}


\subsection{Quality Attributes}

With the given use cases we will then be able to define the major quality attributes that are involved in the 
development of this application. We want those qualities to be measurable and testable so we can verify if the 
system meets the needs our \glsplural{stakeholder} \cite{refbook:DSHC}.

\begin{table}[H]
    \setstretch{1.0}
    \begin{tabularx}{\textwidth}{lcXc}
        \toprule
        ID & Quality Attribute & Scenario & Associated Use Case  \\
        \midrule
        QA-1 & Performance & A \gls{client} register their e-mail address and he can immediate browse in the app. & UC-1 \\
        QA-2 & Performance & A \gls{client} opens the app and he can immediate browse in the app. & UC-2 \\
        QA-3 & Performance & A \gls{client} choose a \gls{provider} and place his order. After the confirmation
        of payment, a push-message is displayed in the app confirming the purchase. & UC-3 \\
        QA-4 & \textit{[to be defined]} & A \gls{client}  register his credit card or select another payment method and the
        confirmation as soon as he confirmed with his \gls{provider}. & UC-4 \\
        QA-5 & Usability & A \gls{provider} is able to register his company, specify the kind of products he offers and upload
        a logo or picture of his shop. & UC-5 \\
        QA-6 & Usability & A \gls{provider} is able to update in the app if he is offering for that day any product. &  UC-6 \\
        QA-7 & Interoperability & A \gls{client} can register his e-mail using another account (Google, Microsoft, Facebook)
        in a federated environment & UC01 \\
        QA-8 & Interoperability & A \gls{client} can pay the order using a \gls{mobile payment gateway} (i.e. Stripe, Square, PayPay, 
        SecurePay) & UC-1 \\
        \bottomrule
    \end{tabularx}
\end{table}

\newpage
The defined quality attributes are represented in the following scenarios:

\begin{table}[H]
    \setstretch{1.0}
    \begin{tabularx}{\textwidth}{|c|X|}
        \hline
        \multicolumn{2}{c}{\textbf{Performance}} \\
        \hline
        \toprule
        \multicolumn{1}{c}{Scenario} & \multicolumn{1}{c}{Value} \\
        \midrule
        Source & \gls{client}  \\
        Stimulus & wishes to create an account \\
        Artifact & platform \\
        Environment & runtime \\
        Response & immediate access to the app \\
        Response Measure & time between confirmation and access \\
         & \\
        Source & \gls{client}  \\
        Stimulus & wants to search fo restaurants or bakeries \\
        Artifact & platform \\
        Environment & peak period, between 6 and 7 pm on Friday \\
        Response & immediate access to the offers \\
        Response Measure & how quick does the \gls{client}  get an updated regarding availability of products \\
        & \\
        Source & \gls{client}  \\
        Stimulus & place an order \\
        Artifact & platform \\
        Environment & peak period, between 6 and 7 pm on Friday \\
        Response & confirmation of the purchase after the payment \\
        Response Measure & time between confirmation of the payment and confirmation of the order \\
        \bottomrule
    \end{tabularx}
\end{table}


\begin{table}[H]
    \setstretch{1.0}
    \begin{tabularx}{\textwidth}{|c|X|}
        \hline
        \multicolumn{2}{c}{\textbf{Usability}} \\
        \hline
        \toprule
        \multicolumn{1}{c}{Scenario} & \multicolumn{1}{c}{Value} \\
        \midrule
        Source & \gls{provider} \\
        Stimulus & wants to offer his remaining products in the app \\
        Artifact & platform \\
        Environment & working time, during afternoon \\
        Response & offer available in the app \\
        Response Measure & How long did the registration and upload process took? Were all necessary information
        available in the app or did the \gls{provider} need to search it outside the app? How long did the registration
        process took? \\
         & \\
        Source & Registered \gls{provider} \\
        Stimulus & wants wants to make a last minute offer \\
        Artifact & platform \\
        Environment & peak period, between 6 and 7 pm on Friday \\
        Response & immediate availability of the offer in the app \\
        Response Measure & how long did it take for the \gls{provider} to upload the offer? Was it easy to input all
        necessary information like, quantity, location and take-away time? Can he do it without any burden? \\
        \bottomrule
    \end{tabularx}
\end{table}

\newpage
\subsection{Constraints}

In general, we can say that constrains are burdens to the development of the project. They define a set of non-negotiable
rules that must be exist \cite{refonline:EFAD}. 

In this project we must distinguish between Technical and Business Constraints. The former describes specific elements
of the project, like programming language, released platform (i.e. operational systems) and technical decisions related to 
the functionalities. The latter deals with management elements\cite{refonline:EFAD} such time, budget and team.


\begin{table}[htb]
    \setstretch{1.0}
    \begin{tabularx}{\textwidth}{lclX}
        \toprule
        ID & Constraint & Category & Description \\
        \midrule
        CT-1 & Programming Language & Technical & Java, Kotlin, iOS, Swift \\
        CT-2 & Platform & Technical & Android, IoS \\
        CT-3 & Payment & Technical & Creating own framework or integrating with existing one (Google Pay, Apple Pay, PayPall) \\
        CT-4 & Login & Technical & Using or not federation or creating own login system  \\
        CT-5 & Time to first prototype release & Business & How long until a first prototype that can be tested wir real users  \\
        CT-6 & Testing time & Business & Time window to test general acceptance \\
        CT-7 & Budget & Business & To maintain a team during the testing phase \\
        CT-8 & Team & Business & To analyze the main usage of the app for further development \\
        \bottomrule
    \end{tabularx}
\end{table}







%https://medium.com/@janerikfra/architectural-drivers-in-modern-software-architecture-cb7a42527bf2 (important)
% https://www.ecs.csun.edu/~rlingard/COMP684/Example2SoftArch.htm
% https://upcommons.upc.edu/bitstream/handle/2099.1/18373/90629.pdf

% Registration Process for Clients QA1
% Registration Process for Restaurants/Backery QA2
% Login Clients QA3
% Login Restaurant/Backery QA4
% Upload offers QA6
% Purchase QA6
% Receive Confirmation QA7

% Performance
% After upload offer, how long until displyed QA8

% Usuability
% Registration for Client: using existing account or Name/email
% Registration for Restaurant/bakery: Upload name, picture, location and type of products

% Purchase client: choose available RA-options
% Payment: register card, use existing accounts (paypal/google pay)

% Upload offer RA: registered RAs activate that they are offering
% Receive order: after payment confirmed, RA recieves order


% Availability

% Modifiability

    %\section{4+1 Architectural View Model}


In this section we will describe the \gls{app} using the 4+1 Architectural View Model. With this model we will represent
the \gls{app} using five different views, which should focus on specific elements of the project. Each view provide
a different purpose \cite{refart:KR41}. For this project we will provide the 3 following views of the 4+1 Architectural View 
Model:

\begin{itemize}
    \item \textbf{Scenario view}: simple description for the end user 
    \item \textbf{Behaviour view}: description of the existing processes
    \item \textbf{Structural view}: object-oriented decomposition
\end{itemize}

The scenario view was presented in the section \ref{Primary_Functionality} of this project.

\newpage
\subsection{Behaviour view}
The following \gls{activity diagram} depicts the register and login procedure within the app.

\begin{figure}[H]
    \centering
    \includegraphics[width=1\textwidth]{assets/login_AC.jpg}
    \caption{Login procedures}
    \label{fig:login_register}
\end{figure}

\newpage
\subsection{Structural view}
To describe this view we choose a \gls{class diagram}. With it we may provide a static description of elements
within the structure of our system. They can also be used during the programming process to display what is needed
to be done.

\begin{figure}[H]
    \centering
    \includegraphics[width=1\textwidth]{/home/bruno/git/Soft.Arch/assets/classes_CD.jpg}
    \caption{Classes of the project}
    \label{fig:class_CD}
\end{figure}
 



    \section{Architecture Constraints}

In this project we must distinguish between Technical (CT-T-\#) and Business (CT-B-\#) Constraints. The former describes specific elements
of the project, like programming language, released platform (e.g. operational systems) and technical decisions related to 
the functionalities. The latter deals with management elements [\cite{refonline:EFAD}] (e.g time, budget and team). The following
tables describes the technical and the business constraints of this project: 

\begin{table}[H]
    \setstretch{1.0}
    \begin{tabularx}{\textwidth}{|c|c|X|}
        \hline
        \multicolumn{3}{c}{\textbf{Technical}} \\
        \hline
        \toprule
        \multicolumn{1}{c}{Id} & \multicolumn{1}{c}{Constraint} & \multicolumn{1}{c}{Description} \\
        \midrule
        CT-T-1 & Programming Language & A multilanguage (Java, Kotlin, iOS, Swift) approach increases
        the maintainability burden and consequently the costs (see CT-B-4). It can also interfere with
        compatibility with different kind of device.s  \\ 
        CT-T-2 & Platform & Offering the application for different platforms (iOS and/or Android) increases
        costs for maintainability and requires a bigger team. Since the prototype should run during the
        first year mainly to gather information about consumer behavior the costs in this test phase can
        increase rapidly if we decide to develop for the most common platforms. \\ 
        CT-T-3 & Payment & One the one hand creating an own payment framework can gives full control of the application,
        but on the other hand it will required specialized team and increases costs and time (see CT-B-4). \\
        CT-T-4 & Payment gateway & Using existing \gls{mobile payment gateway} reduces development time, but demands
        fully Interoperability of the app with the existing gateways. It may also be a problem if the \gls{client}
        don't use this kind of payment method. \\
        CT-T-5 & Login & Using existing \gls{federated login} decreases development time, but like CT-T-4 demands
        fully interoperability of the app with appliances. It may also be a problem if the \gls{client}
        don't trust this kind of login. \\
        \bottomrule
    \end{tabularx}
\end{table}

\begin{table}[H]
    \setstretch{1.0}
    \begin{tabularx}{\textwidth}{|c|c|X|}
        \hline
        \multicolumn{3}{c}{\textbf{Business}} \\
        \hline
        \toprule
        \multicolumn{1}{c}{Id} & \multicolumn{1}{c}{Constraint} & \multicolumn{1}{c}{Reasoning} \\
        \midrule
        CT-B-1 & Time to first prototype release & How much time is acceptable from starting the project
        until we have a functional prototype that can be used by our user? \\
        CT-B-2 & Development Team & The existing team can cover the main existing platforms, but their availability
        may be restricted to due work on other projects Specially for the maintainability of the app it can represents
        a problem. \\
        CT-B-3 & Analytical Team & During running phase of the prototype it will be necessary to have a team in charge
        of evaluating and interpreting the collected data, to find out if the goals are being achieved. \\
        CT-B-4 & Budget & Since this application falls in the category ```middle app'' according to [\cite{refonline:SPDLOAD}]
        the available budget of US\$ 150.000 should cover the development of the main functionality and the data analysis 
        (see CT-B-3) \\
        \bottomrule
    \end{tabularx}
\end{table}




%https://medium.com/@janerikfra/architectural-drivers-in-modern-software-architecture-cb7a42527bf2 (important)
% https://www.ecs.csun.edu/~rlingard/COMP684/Example2SoftArch.htm
% https://upcommons.upc.edu/bitstream/handle/2099.1/18373/90629.pdf

% Registration Process for Clients QA1
% Registration Process for Restaurants/Backery QA2
% Login Clients QA3
% Login Restaurant/Backery QA4
% Upload offers QA6
% Purchase QA6
% Receive Confirmation QA7

% Performance
% After upload offer, how long until displyed QA8

% Usuability
% Registration for Client: using existing account or Name/email
% Registration for Restaurant/bakery: Upload name, picture, location and type of products

% Purchase client: choose available RA-options
% Payment: register card, use existing accounts (paypal/google pay)

% Upload offer RA: registered RAs activate that they are offering
% Receive order: after payment confirmed, RA recieves order


% Availability

% Modifiability

    \section{Context and Scope}

Since this system relies on the correct working of external elements it is important that their interaction is 
corrected displayed.

\subsection{Business Context}

\begin{figure}[H]
    \centering
    \includegraphics[width=0.5\textwidth]{assets/business_context.jpg}
    %\caption{Business Context}
    \label{fig:business_context}
\end{figure}

\begin{table}[H]
    \setstretch{1.0}
    \begin{tabularx}{\textwidth}{lX}
    \toprule
    Artefact & Description   \\
    \midrule
    \gls{client} & Searches for a last time offer from a restaurant, bakery or pastry. \\
    \gls{provider} & Offers a still consumable product that was not sold during normal working time. \\
    Payment & Deals with the payment processing using registered information from another payment platforms. \\
    Login/Registration & Authenticated \glsplural{user} using logins from other platforms.  \\
    \bottomrule
    \end{tabularx}
\end{table}

\subsection{Technical Context}

\begin{figure}[H]
    \centering
    \includegraphics[width=0.5\textwidth]{assets/technical_context.jpg}
    %\caption{Technical Contes}
    \label{fig:technical_context}
\end{figure}

% \begin{table}[H]
%     \setstretch{1.0}
%     \begin{tabularx}{\textwidth}{lX}
%     \toprule
%     Artefact & Description \\
%     \midrule
%     Graphical Interface & \gls{client} and \gls{provider} have an own interface to interact. \gls{provider} can
%     access view their offer also with a \gls{client}'s perspective. \\
%     Offer Database & \glsplural{client} and \glsplural{provider} can make requests to the database to inquire
%     about its content. \\
%     \gls{api} & For login and payment the authentication and authorization take places on the external service. \\
%     \bottomrule
%     \end{tabularx}
% \end{table}

\begin{table}[H]
    \setstretch{1.0}
    \begin{tabularx}{\textwidth}{lX}
    \toprule
    Artefact & Description \\
    \midrule
    Graphical Interface & \gls{client} and \gls{provider} have an own interface to interact. \gls{provider} can
    access view their offer also with a \gls{client}'s perspective. \\
    Offer Database & \glsplural{client} and \glsplural{provider} can make requests to the database to inquire
    about its content. \\
    \gls{api} & For login and payment the authentication and authorization take places on the external service. \\
    \bottomrule
    \end{tabularx}
\end{table}
    %\section{Solution and Strategy}

include
    \section{Building Block View}

In this section we will describe the \gls{app} using the 4+1 Architectural View Model. With this model we will represent
the \gls{app} using five different views, which should focus on specific elements of the project. Each view provide
a different purpose \cite{refart:KR41}. For this project we will provide the 3 following views of the 4+1 Architectural View 
Model:

\begin{itemize}
    \item \textbf{Scenario view}: simple description for the end user 
    \item \textbf{Behaviour view}: description of the existing processes
    \item \textbf{Structural view}: object-oriented decomposition
\end{itemize}

The scenario view was presented in the figure \ref{fig:preliminary_use_case} of this project.

\subsection{Behaviour view}
The following \gls{activity diagram} depicts the register and login procedure within the app.

\begin{figure}[H]
    \centering
    \includegraphics[width=0.7\textwidth]{assets/login_AC.jpg}
    \caption{Login procedures}
    \label{fig:login_register}
\end{figure}

\subsection{Structural view}
To describe this view we choose a \gls{class diagram}. With it we may provide a static description of elements
of our app. This will be very relevant for the developing process of the \gls{app}.

The first part of the this diagram describes the element within the \gls{provider}. It contains one or more addresses and it 
can offer one or more products. A provider will also fall into the category restaurant, bakery or pastry.

\begin{figure}[H]
    \centering
    \includegraphics[width=0.7\textwidth]{assets/Provider_Addr_Item.jpg}
    \caption{Provider overview}
    \label{fig:Provider_addr_item}
\end{figure}
 
The class dedicated to the \glsplural{client} should be as simple as possible. It should provide basic interaction like
registering, logging, deleting account, viewing product and placing order. The two last actions will stablish the communication 
with the \glsplural{provider}.

\begin{figure}[H]
    \centering
    \includegraphics[width=0.7\textwidth]{assets/client_CD.jpg}
    \caption{Client Overview}
    \label{fig:client_CD}
\end{figure}

Finally we have an order placed by a \gls{client} and processed by a \gls{provider}. Here we will rely on a third party 
to stablish the payment procedures.

\begin{figure}[H]
    \centering
    \includegraphics[width=0.7\textwidth]{assets/order_cd.jpg}
    \caption{Order Overview}
    \label{fig:order_cd}
\end{figure}

This final graphic show the whole classes in combination:

\begin{figure}[H]
    \centering
    \includegraphics[width=0.7\textwidth]{assets/classes_CD.jpg}
    \caption{Classes Overview}
    \label{fig:class_CD}
\end{figure}

 

    %\include{6_Runtime_View}
    %\include{7_Deployment_View}
    %\section{Crosscutting Concept}

In this chapter we will present the technical solutions that we will use to develop this project.
For each quality attribute we will present the chosen tactitcs.

\subsection{Solution for Usability}

\subsection{Solution for Interoperability}

The communication with the 3rd party components should during the whole lifetime of the App reliable. Since we are dealing with
two different services, \gls{mobile payment gateway} and \gls{federated login}, we will describe the integration processes 
according to each specification.

From the third party applications we expect the following interaction:

\begin{figure}[H]
    \centering
    \includegraphics[width=0.7\textwidth]{assets/sequence_login_payment.jpg}
    \caption{Sequence of actions with 3rd party applications}
    \label{fig:sequence_login_payment}
\end{figure}

\subsubsection{Payment Gateway}

The usage of \gls{mobile payment gateway} offers three possibilities \cite{refonline:ZOPG}:

\begin{itemize}
    \item Redirection to payment processor's page
    \item Payment data and processing inside the application
    \item Payment data entered in the app, but processed with an \acrshort{api}
\end{itemize}

The third option stays in direct contact with our top quality attribute, usability. Since we want to offer a easy shopping
experience, the payment process should also be harmonic with other features.

\begin{table}[H]
    \setstretch{1.0}
    \begin{tabularx}{\textwidth}{|c|c|X|}
        \toprule
        \multicolumn{1}{c}{Tactict} & \multicolumn{1}{c}{Pattern} & \multicolumn{1}{c}{Motivation} \\
        \midrule
        \textbf{Limit Dependencies} & \gls{wrapper} & The \gls{api} will be the intermediary for the payment process. For the 
        \glsplural{client} all visible steps will occur in the app, without being sent to another page. On the background
        the \gls{api} will receive the input and send it to the payment gateway. The verification takes place in gateway, 
        which then communicate with the financial institute of the client and send the payment to the \gls{provider} 
        \cite{refonline:ZOPG}.  \\
        \bottomrule
    \end{tabularx}
\end{table}

\subsubsection{Federated Authentication}

Using of \gls{federated login} reduces burden of saving user credentials locally. It also improves the Usability so users
do not have to create and remember another username and password. The authentication process takes place on the third 
party operator, as seen in the picture \ref{fig:sequence_login_payment}. 

\begin{table}[H]
    \setstretch{1.0}
    \begin{tabularx}{\textwidth}{|c|c|X|}
        \toprule
        \multicolumn{1}{c}{Tactict} & \multicolumn{1}{c}{Pattern} & \multicolumn{1}{c}{Motivation} \\
        \midrule
        \textbf{\gls{microservice}} & \gls{API Gateway} & It increases security, so the microservice is not directly
        exposed to the external world. It reduces the complexity of the microservice, since the gateway will have to deal
        with data transfer rate, tokens and other activities. Dealing with failures would also be handled and logged
        by the microservice \cite{refonline:javtop}.\\
        \bottomrule
    \end{tabularx}
\end{table}

\subsection{Solution for Performance}

We want our app to have a fast (no more than 1 second) response time. By clicking on an offer a \gls{client} should
have it immediately displayed on his/her screen. Updated made by \glsplural{provider} should also be promptly available
for \glsplural{client} to browse.

\begin{table}[H]
    \setstretch{1.0}
    \begin{tabularx}{\textwidth}{|c|c|X|}
        \toprule
        \multicolumn{1}{c}{Tactict} & \multicolumn{1}{c}{Pattern} & \multicolumn{1}{c}{Motivation} \\
        \midrule
        \textbf{Increase Resources} & \gls{Load Balancer} & Specially during peak times we want our users to have a 
        smoothly and fast interaction with the app. \glsplural{provider} and \glsplural{client} should perform their
        tasks, either browsing, purchasing or uploading offering without having to wait to get a response. With this
        decision all requests would be forwarded to the server that are available avoiding queuing of requests. \\
        \bottomrule
    \end{tabularx}
\end{table}
    %\section{Architectural Decisions Records}

In this section we will present the motivation to our decisions within this project.

\subsection{Chosen Preliminary Functionality}

\subsection{Highlighted Constraints}

\subsection{Stakeholders}

\subsection{Stakeholders}

\subsection{Stakeholders}

\begin{table}[H]
    \setstretch{1.0}
    \begin{tabularx}{\textwidth}{lX}
    \toprule
    ID & Reasoning   \\
    \midrule
    F-1 & Owner of a restaurant, bakery or pastry. \\
    F-2 & Person who wants to buy last minute products from a provider. \\
    F-3 & Team in charge of creating the application using existing tactics and creating new solutions. \\
    F-5 & ``Clean Up the Word (R)'' & Members of the  \\
    F-7 & Part of the society who aims to find environmental solutions to daily problems. \\
    F-8 & Part of the society who aims to find environmental solutions to daily problems. \\
    \bottomrule
    \end{tabularx}
\end{table}

include

describe each decision:
why this tactic/pattern
    \section{Quality Requirements}

\subsection{Quality Tree}

The priority of each element will be expressed using the following notation:

\begin{itemize}
    \item ([Customer view], [Architect view])
    \item H - High
    \item M - Medium
    \item L - Low
\end{itemize}

\Glsplural{user} and the development team have different perspective of an app. The former think about how attractive and easy to use
it is, the latter want to build something what achieves a goal. For that reason is the interpretation of the priority sometimes
so different, according to which group has been asked.

\begin{table}[H]
    \setstretch{1.0}
    \begin{tabularx}{\textwidth}{lXX}
        \toprule
        ID & For \Glsplural{user} & For Development Team  \\
        \midrule
        QA-1 & All important information should be there so his shop can be well promoted & 
        An initial registration with filter for the input is important, but estetical details are
        the goal now. \\
        QA-2 & Once he get something new, he wants to make it available & First we need to guarantee that no overrides occur
        than they see if it is promptly displayed. \\
        QA-3 & They want to browse and see all available options & Search engine can be very helpful, 
        but filtering can wait a litte, since it does not affect the app itself \\
        QA-4 & The most important is that they can use and purchase & This integration must be done fast and careful
        so no mistakes shows up. \\
        QA-5 & They just want to easily and secure pay, it does not matter how it works & The compliance with payment regulations is a must, since any mistake can costs huge fines
        and damage to the image of the company. \\
        QA-6 & They don't want to wast time with loading pages & The loading time can be fixed once there is a structure that allows
        loading in the first place.  \\
        QA-7 & They want to get confirmation that everything worked fine. & The communication between the \gls{API} and the payment provider show comply with all existing regulations. 
        Push notification can be added once the main feature works. \\
        QA-8 & That is something that they don't want to see, but want to make sure that it exists & Since the payment is processed by the third party operator, all concerns should be addressed to them
        and specified in the \glsfirst{SLA} \\
        \bottomrule
    \end{tabularx}
\end{table}



\begin{figure}[H]
    \centering
    \includegraphics[width=0.7\textwidth]{assets/quality_tree.png}
    %\caption{Quality Tree}
    \label{fig:quality_tree}
\end{figure}


\subsection{Evaluation Scenarios} 
From the requirements, \ref{Requirement_Overview}, we could develop the following uses cases and depict the main quality 
attributes of this project. 

\begin{table}[H]
    \setstretch{1.0}
    \begin{tabularx}{\textwidth}{lX}
    \toprule
    Use Case & Description  \\
    \midrule
    UC-1: Register as \gls{client} & The \gls{client} registers an e-mail address.\\
    UC-2: Login & The \gls{client} logins in to the system. \\
    UC-3: Places an order & The \gls{client} chooses a \gls{provider}. \\
    UC-4: Register payment & The \gls{client} registers a payment method. \\
    UC-5: Register as \gls{provider} & The \gls{provider} registers their facility and products. \\
    UC-6: Update availability & The \gls{provider} uploads their product catalog. \\
    \bottomrule
    \end{tabularx}
\end{table}

With the following use cases we will  be able to define the major quality attributes that are involved in the 
development of this application. They should be measurable and testable so we can verify if the system meets 
the needs our \glsplural{stakeholder} \cite{refbook:DSHC}.

\begin{table}[H]
    \setstretch{1.0}
    \begin{tabularx}{\textwidth}{lcXc}
        \toprule
        ID & Quality Attribute & Scenario & Associated Use Case  \\
        \midrule
        QA-1 & Usability & A \gls{provider} is able to register his company, to specify the kind of products he offers 
        and upload a logo or picture of his shop and products in a easy and fast (within 5 Minutes) fashion. & UC-5 \\
        QA-2 & Usability & A \gls{provider} is able to update the offers at any time. &  UC-6 \\
        QA-3 & Usability & A \gls{client} is able to search and filter options. &  UC-6 \\
        QA-4 & Interoperability & A \gls{client} can register his e-mail using another account (Google, Microsoft, Facebook)
        in a \gls{federated login} & UC-1 \\
        QA-5 & Interoperability & A \gls{client} can pay the order using a \gls{mobile payment gateway} (e.g. Stripe, Square, PayPay, 
        SecurePay) & UC-4 \\
        QA-5 & Performance & A \gls{client} registers his/her e-mail address and can immediately browse in the app. & UC-1 \\
        QA-6 & Performance & A \gls{client} opens the app and he can immediately search for products or \glsplural{provider}. & UC-2 \\
        QA-7 & Performance & A \gls{client} chooses a \gls{provider} and places his order. After the confirmation
        of payment, a push-message is displayed in the app confirming the purchase. & UC-3 \\
        QA-8 & Security & The payment process should be secure and within the app. It should also give the \gls{client} the feeling
        of security. The \gls{client} inserts his payment information it is processed by the payment operator. & UC-4 \& QA-5 \\
        \bottomrule
    \end{tabularx}
\end{table}

\newpage
The defined quality attributes are represented in the following scenarios:

\begin{table}[H]
    \setstretch{1.0}
    \begin{tabularx}{\textwidth}{|c|X|X|X|}
        \hline
        \multicolumn{4}{c}{\textbf{Usability}} \\
        \hline
        \toprule
        \multicolumn{1}{|c|}{\textbf{Scenario}} & \multicolumn{3}{|c|}{\textbf{Value}} \\
        \midrule
        Source & \Gls{provider} & Registered \Gls{provider} & \Gls{client}  \\
        \hline
        Stimulus & wants to register his shops & wants wants to make a last minute offer & wants to search/filter offers \\
        \hline
        Artifact & app & app & app \\
        \hline
        Environment & working time, during afternoon & peak period, between 4 and 7 pm on Friday & peak period, between 4 and 7 pm on Friday \\
        \hline
        Response & offer available in the app & immediate availability of the offer in the app & display of the filter/search output \\
        \hline
        Response Measure & How long did the registration and upload process take? How many and what kind of error messages did the \gls{provider} get?
        & How long did it take to upload an offer? How many and what kind of error messages did the \gls{provider} get? 
        & What kind of inputs did the user has to place until he finds what he wants? Did he have to type anything or were filter/search
        options available? How long it takes until the client finds a product? \\
        \bottomrule
    \end{tabularx}
\end{table}


% \begin{table}[H]
%     \begin{tabularx}{\textwidth}{|c|X|}
%         \hline
%         \multicolumn{2}{c}{\textbf{Usability - BACKUP}} \\
%         \hline
%         \toprule
%         \multicolumn{1}{c}{Scenario} & \multicolumn{1}{c}{Value} \\
%         \midrule
%         Source & \gls{provider} \\
%         Stimulus & wants to register his shops \\
%         Artifact & app \\
%         Environment & working time, during afternoon \\
%         Response & offer available in the app \\
%         Response Measure & How long did the registration and upload process take? How many and what kind of error messages
%         did the \gls{provider} get?\\
%          & \\
%         Source & Registered \gls{provider} \\
%         Stimulus & wants wants to make a last minute offer \\
%         Artifact & app \\
%         Environment & peak period, between 4 and 7 pm on Friday \\
%         Response & immediate availability of the offer in the app \\
%         Response Measure & How long did it take to upload an offer? How many and what kind of error messages did the 
%         \gls{provider} get? \\
%         & \\
%         Source & Registered \gls{client} \\
%         Stimulus & wants to search/filter offers \\
%         Artifact & app \\
%         Environment & peak period, between 4 and 7 pm on Friday \\
%         Response & display of the filter/search output \\
%         Response Measure & What kind of inputs did the user has to place until he finds what he wants?
%         Did he have to type anything or were filter/search options available? How long it takes until the client
%         finds a product? \\
%         \bottomrule
%     \end{tabularx}
% \end{table}

 \begin{table}[H]
    \setstretch{1.0}
    \begin{tabularx}{\textwidth}{|c|X|X|}
        \hline
        \multicolumn{3}{c}{\textbf{Interoperability}} \\
        \hline
        \toprule
        \multicolumn{1}{|c|}{\textbf{Scenario}} & \multicolumn{2}{|c|}{\textbf{Value}} \\
        \midrule
        Source & \Gls{client} & \Gls{client}  \\
        \hline
        Stimulus & wants register using a \gls{federated login} & wants to pay using existing mobile payment account \\
        \hline
        Artifact & app and \gls{federated login} provider & app and \gls{mobile payment gateway} \\
        \hline
        Environment & peak period (on the context of the \gls{federated login} provider) & peak period (on the context of the gateway) \\
        \hline
        Response & authentication succeed or failed & confirmation / declined \\
        \hline
        Response Measure & How much data was transmitted and how much was queued? Focus on System overload \cite{refart:MKMS}
        & Total amount generated data in the app that are transferred and processed and rejected by the gateway? Focus o connectivity 
        and system overload \cite{refart:MKMS} \\
        \bottomrule
    \end{tabularx}
\end{table}

% \begin{table}[H]
%     \begin{tabularx}{\textwidth}{|c|X|}
%         \hline
%         \multicolumn{2}{c}{\textbf{Interoperability - Backup}} \\
%         \hline
%         \toprule
%         \multicolumn{1}{c}{Scenario} & \multicolumn{1}{c}{Value} \\
%         \midrule
%         Source & \gls{client} \\
%         Stimulus & wants register using a \gls{federated login} \\
%         Artifact & app and \gls{federated login} provider  \\
%         Environment & peak period (on the context of the \gls{federated login} provider)\\
%         Response & authentication succeed or failed\\
%         Response Measure & How much data was transmitted and how much was queued? \\
%         Focus & System overload \cite{refart:MKMS} \\
%         & \\
%         Source & \gls{client} \\
%         Stimulus & wants to pay using existing mobile payment account \\
%         Artifact & app and \gls{mobile payment gateway}  \\
%         Environment & peak period (on the context of the gateway)\\
%         Response & confirmation / declined \\
%         Response Measure & Total amount generated data in the app that are transferred and processed and rejected
%         by the gateway \\
%         Focus & Connectivity and System overload \cite{refart:MKMS} \\
%         \bottomrule
%     \end{tabularx}
% \end{table}

\begin{table}[H]
    \setstretch{1.0}
    \begin{tabularx}{\textwidth}{|c|X|X|X|}
        \hline
        \multicolumn{4}{c}{\textbf{Performance}} \\
        \hline
        \toprule
        \multicolumn{1}{|c|}{\textbf{Scenario}} & \multicolumn{3}{|c|}{\textbf{Value}} \\
        \midrule
        Source & \Gls{client} & \Gls{client} & \Gls{client}  \\
        \hline
        Stimulus & wishes to create an account & wants to search for a \gls{provider} & places an order \\
        \hline
        Artifact & app & app & app \\
        \hline
        Environment & weekend between 3 and 7 PM & peak period, between 6 and 7 pm on a Friday & peak period, between 6 and 7 pm on a Friday \\
        \hline
        Response & immediate access to the app  & immediate access to the offers  & confirmation of payment / payment declined \\
        \hline
        Response Measure & time between confirmation and access & how quickly does the client's device get update of availabilities 
        & How long did take until the client get the confirmation/declined of payment? \\
        \bottomrule
    \end{tabularx}
\end{table}


% \begin{table}[H]
%     \begin{tabularx}{\textwidth}{|c|X|}
%         \hline
%         \multicolumn{2}{c}{\textbf{Performance - Backup}} \\
%         \hline
%         \toprule
%         \multicolumn{1}{c}{Scenario} & \multicolumn{1}{c}{Value} \\
%         \midrule
%         Source & \gls{client}  \\
%         Stimulus & wishes to create an account \\
%         Artifact & app \\
%         Environment & weekend between 3 and 7 PM \\
%         Response & immediate access to the app \\
%         Response Measure & time between confirmation and access \\
%          & \\
%         Source & \gls{client}  \\
%         Stimulus & wants to search for a \gls{provider} \\
%         Artifact & app \\
%         Environment & peak period, between 6 and 7 pm on a Friday \\
%         Response & immediate access to the offers \\
%         Response Measure & how quickly does the client's device get update of availabilities \\
%         & \\
%         Source & \gls{client}  \\
%         Stimulus & places an order \\
%         Artifact & platform \\
%         Environment & peak period, between 6 and 7 pm on a Friday \\
%         Response & confirmation of payment / payment declined \\
%         Response Measure & How long did take until the client get the confirmation/declined of payment?\\
%         \bottomrule
%     \end{tabularx}
% \end{table}

\begin{table}[H]
    \setstretch{1.0}
    \begin{tabularx}{\textwidth}{|c|X|X|}
        \hline
        \multicolumn{3}{c}{\textbf{Security}} \\
        \hline
        \toprule
        \multicolumn{1}{|c|}{\textbf{Scenario}} & \multicolumn{2}{|c|}{\textbf{Value}} \\
        \midrule
        Source & \Gls{client} & \Gls{client} \\
        \hline
        Stimulus & clicks on registration using an existing login & click on pay using an existing mobile payment account \\
        \hline
        Artifact & app, \gls{API Gateway} and \gls{federated login} provider & app, \gls{microservice} and \gls{mobile payment gateway} \\
        \hline
        Environment & peak period (on the context of the \gls{federated login} provider) & peak period (on the context of the gateway) \\
        \hline
        Response & authentication succeed or failed & confirmation / declined \\
        \hline
        Response Measure & Required time and effort to intercept and/or block requests (create \glsfirst{DoS})
        & Extension to image damage of the app and of the company in case of attack \\
        \bottomrule
    \end{tabularx}
\end{table}
    %\section{Risk and Technical Debt}
    %\include{12_Glossary}
    \nocite{*}
    \bibliography{reference}
    \bibliographystyle{apalike}
        
\end{document}