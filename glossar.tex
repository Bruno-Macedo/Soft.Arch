\newacronym{fao}{FAO}{Food and Agriculture Organization of the United Nations}
\newacronym{un}{UN}{United Nations}
\newacronym{api}{API}{Application Programming Interface}

\newglossaryentry{user} {
    name={User},
    plural={users},
    description={See stakeholder}
}

\newglossaryentry{app} {
    name={App},
    description={It refers to the mobile application to be developed}
}
\newglossaryentry{stakeholder} {
    name={Stakeholder},
    plural={stakeholders},
    description={Describes all kind of potential person or entity that may have interest using the app}
}
\newglossaryentry{client} {
    name={Client},
    plural={clients},
    description={Since we have two major stakeholders that will use the app, the word client will
    specify the one that places an order in the app}
}
\newglossaryentry{provider} {
    name={Provider},
    plural={providers},
    description={The second major stakeholders are those who offer their products. They can be
                restaurants, bakeries, pastries and similar}
}
\newglossaryentry{upload} {
    name={Upload},
    description={Will designate the act of pushing textual or visual content into the app. It will
                be mostly done by a provider}
}
\newglossaryentry{register} {
    name={Register},
    description={Describes the act of inputting name, e-mail, company name, product or any other kind of
    textual data used to identify a stakeholder or a product}
}
\newglossaryentry{activity diagram} {
    name={Activity Diagram},
    description={This kind of diagram shows the behavior of a system, it depicts in a graphical fashion 
    the logic of a single use case \cite{refinbook:Baresi2009}}
}
\newglossaryentry{class diagram} {
    name={Class Diagram},
    description={This kind of diagram presents the structure of a system with its classes, attributes,
    methods and relationships\cite{refonline:IBMCD}}
}
\newglossaryentry{use case diagram} {
    name={Use Case Diagram},
    description={This kind of diagram presents the main requirements and functionalitiesPa of a systems. It displays
    a simplified overview of core purpose of the application \cite{refart:YWRUS}}
}
\newglossaryentry{mobile payment gateway} {
    name={Mobile Payment Gateway},
    description={Those services works as an intermediary between customer, merchant and bank/credit card company. 
    Here a payment request is sent to the gateway and forwarded to the approval instances. The core functionality 
    of those gateways is the cryptography within the communication steps\cite{refonline:VPGI}}
}

\newglossaryentry{federated login} {
    name={Federated Login},
    description={Authentication method in which users use existing accounts to gain access to another domains or systems
    without the need of creating new credentials. The authenticity of a user is attested by service and granted to another 
    \cite{refonline:MRFL}}
}
\newglossaryentry{system response} {
    name={System Response},
    description={The output of a system after an input \cite{refonline:HWHE}.
    \cite{refonline:MRFL}}
}
\newglossaryentry{API} {
    name={Application Programming Interface - API},
    plural={APIs},
    description={Software intermediary that promotes the communication between different systems/applications/
    softwares\cite{refonline:MSAPI}}
}

\newglossaryentry{wrapper} {
    name={Wrapper},
    plural={wrappers},
    description={Element used to encapsulate the complexity of one entity so it can be processed by another entity 
    \cite{refonline:techwrap}}
}

\newglossaryentry{microservice} {
    name={Microservice},
    plural={microservices},
    description={Software architecture approach made of small independent services used to communicate with other
    resources like APIs. The advantage of using this architecture is its scalability and maintainability, since 
    each service is responsible for a very small group of correlated tasks \cite{refonline:awsmicro}}
}

\newglossaryentry{API Gateway} {
    name={API Gateway},
    plural={API Gateways},
    description={Server used an single entry point into a system. It forwards requests to the used service. It is 
    broadly used for authentication, auditing and logging services \cite{refonline:crpag}}
}

\newglossaryentry{Load Balancer} {
    name={Load Balancer},
    description={Device used to distribute traffic/resource/request across different servers, so one of them
    are not overloaded \cite{refonline:nglb}}
}

\newglossaryentry{bots} {
    name={bots},
    description={Short for robot. It is a software that performs automated and repetitive tasks. It is usually
    controlled by a malicious actor who targets a network or service. They can be used consume resources
    and make a service unavailable, steal credentials and other attacks \cite{refonline:kpbot}}
}

\newglossaryentry{Intercepting Validator} {
    name={Intercepting Validator},
    description={Mechanism used to check if user's input (data or file) corresponds to the criteria defined in
    the app. Non-conform input are discarded before even reaching the app \cite{refonline:kpbot}}
}

\newglossaryentry{risk assessment} {
    name={Risk Assessment},
    description={Report used to identify risk/weakness that may exist in a project. \cite{refclas:smvkl11}}
}

% here goes to the description
\newglossaryentry{DoSS}{
    name=\glslink{DoS}{Denial of Service (\gls{DoS})},
    description={Intentional interruption or laming of network services}
}

% here goes to the acronym
\newglossaryentry{DoS}{
    type=\acronymtype,
    name=DoS,
    first={Denial of Service (DoS)\glsadd{DoSS}},
    see=[Glossary:]{\gls{DoSS}}, 
    description=\glslink{DoSS}{Denial of Service}
}

% here goes to the description
\newglossaryentry{SLA2}{
    name=\glslink{SLA}{Service Level Agreement (\gls{SLA})},
    description={Contract between a service provider and a consumer regarding the expected level of service that is accepted.
    This contract is based on defined metrics, measures to remediate occurring problems and penalties if the contracted
    level is not achieved \cite{refmisc:SOSLA} }
}

% here goes to the acronym
\newglossaryentry{SLA}{
    type=\acronymtype,
    name=SLA,
    first={Service Level Agreement (SLA)\glsadd{SLA2}},
    see=[Glossary:]{\gls{SLA2}}, 
    description=\glslink{SLA2}{Service Level Agreement}
}


