\section{Constraints}

In this project we must distinguish between Technical (CT-T-\#) and Business (CT-B-\#) Constraints. The former describes specific elements
of the project, like programming language, released platform (e.g. operational systems) and technical decisions related to 
the functionalities. The latter deals with management elements [\cite{refonline:EFAD}] (e.g time, budget and team). The following
tables describes the technical and the business constraints of this project: 

\begin{table}[H]
    \setstretch{1.0}
    \begin{tabularx}{\textwidth}{|c|c|X|}
        \hline
        \multicolumn{3}{c}{\textbf{Technical}} \\
        \hline
        \toprule
        \multicolumn{1}{c}{Id} & \multicolumn{1}{c}{Constraint} & \multicolumn{1}{c}{Reasoning} \\
        \midrule
        CT-T-1 & Programming Language - Kotlin & A multilanguage (Java, Kotlin, iOS, Swift) approach increases
        the maintainability burden and consequently the costs (see CT-B-4). It can also interfere with
        compatibility with different kind of device.s  \\ 
        CT-T-2 & Platform - Android & Offering the application for different platforms (iOS and/or Android) increases
        costs for maintainability and requires a bigger team. Since the prototype should run during the
        first year mainly to gather information about consumer behavior the costs in this test phase can
        increase rapidly if we decide to develop for the most common platforms. \\ 
        CT-T-3 & Payment & One the one hand creating an own payment framework can gives full control of the application,
        but on the other hand it will required specialized team and increases costs and time (see CT-B-4). \\
        CT-T-4 & Payment gateway & Using existing \gls{mobile payment gateway} reduces development time, but demands
        fully Interoperability of the app with the existing gateways. It may also be a problem if the \gls{client}
        don't use this kind of payment method. \\
        CT-T-5 & Login & Using existing \gls{federated login} decreases development time, but like CT-T-4 demands
        fully interoperability of the app with appliances. It may also be a problem if the \gls{client}
        don't trust this kind of login. \\
        CT-T-6 & Amount of servers & The existing budget allow us to keep only one full operating server to handle
        the circulating data. \\
        \bottomrule
    \end{tabularx}
\end{table}

\begin{table}[H]
    \setstretch{1.0}
    \begin{tabularx}{\textwidth}{|c|c|X|}
        \hline
        \multicolumn{3}{c}{\textbf{Business}} \\
        \hline
        \toprule
        \multicolumn{1}{c}{Id} & \multicolumn{1}{c}{Constraint} & \multicolumn{1}{c}{Reasoning} \\
        \midrule
        CT-B-1 & Time to first prototype release - Maximal 1 year & To stay on the budget the first release should
        be ready within one year after the approval \\
        CT-B-2 & Development Team - 4 Dev + 1 Sys-Admin & Most of the development team members is allocated on
        other projects and cannot be changed. They should be in charge of the development and maintainability. \\
        CT-B-3 & Analytical Team - 1 person & During running phase of the prototype it will be necessary to have someone 
        in charge of evaluating and interpreting the collected data, to find out if the goals are being achieved. \\
        CT-B-4 & Budget & For this kind of project the maximum budget is US\$ 150.000. It should cover the development of 
        the main functionality and the data analysis (see CT-B-3).  ```middle app'' according to [\cite{refonline:SPDLOAD}] \\
        CT-B-5 & Security and Trust of third party resources & Since login and payment will be processed by third party providers,
        \gls{federated login} and \gls{API Gateway}, a lack of security on their side can cause damage to the reputation of 
        our company. \\
        \bottomrule
    \end{tabularx}
\end{table}



% add the not used quality attributes