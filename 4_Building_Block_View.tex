\section{Building Block View}

In this section we will describe the \gls{app} using some elements of the 4+1 Architectural View Model. With this model we
aim to target an understanding of all our main stakeholders.

We will use 4 different views, which should focus on specific elements of the project. Each view provides a different
purpose \cite{refart:KR41}. For this project we will provide the 3 following views of the 4+1 Architectural View 
Model:

\begin{itemize}
    \item \textbf{Scenario view}: simple description for the end user 
    \item \textbf{Structural view}: object-oriented decomposition
    \item \textbf{Behavior view}: description of the existing processes
\end{itemize}

\subsection{Scenario view}

Our first picture \ref{fig:preliminary_use_case} provided our stakeholders a brief presentation of the basic functionalities of our app. Other elements addressed in this view were presented in the chapter where we discussed the business context, \ref{business_context}. 

%The following picture should give them a shallow view of the app and its external components.

%\begin{figure}[H]
%    \centering
%    \includegraphics[width=0.5\textwidth]{assets/%business_context.jpg}
%    \caption{Diagram to describe the business context}
%    \label{fig:business_context}
%\end{figure}

%\begin{table}[H]
%    \setstretch{1.0}
%    \begin{tabularx}{\textwidth}{lX}
%    \toprule
%    Artefact & Description   \\
%    \midrule
%    \gls{client} & Searches for a last time offer from a restaurant, bakery or pastry. \\
%    \gls{provider} & Offers a still consumable product that was not sold during normal working time. \\
%    Payment & Deals with the payment processing using registered information from another payment platforms. \\
%    Login/Registration & Authenticated \glsplural{user} using logins from other platforms.  \\
%    \bottomrule
%    \end{tabularx}
%\end{table}

An example on how each feature of the app should work can be found in our use case in section \ref{table_use_case}.
More elements of this view will be presented while discussing the internal decisions in chapter \ref{Patterns_Tacticts}. 


\subsection{Structural view}

% put only big graphic and use it for on the 4+1 view

% This following graphics are addressed to our technical team. They provide a deeper view of the model shown in
% \ref{fig:business_context}. The once black-boxed elements showed above are now white-boxed so our development
% team becomes a better understanding of the relevant components of the app.

% \begin{figure}[H]
%     \centering
%     \includegraphics[width=0.4\textwidth]{assets/technical_context.jpg}
%     \caption{Technical Context}
%     \label{fig:technical_context}
% \end{figure}

For the following graphic we choose a \gls{class diagram} to provide our development team a further view of the
structural elements of the project. This first class diagram gives a simple description of the classes that can
exist in the app.

\begin{figure}[H]
    \centering
    \includegraphics[width=0.6\textwidth]{assets/simple_classes_CD.jpg}
    \caption{Level 1 - Class Diagramm}
    \label{fig:simple_class_diagram}
\end{figure}

% The first part of the this diagram describes the element within the \gls{provider}. It contains one or more addresses and it 
% can offer one or more products. A provider will also fall into the category restaurant, bakery or pastry.

% \begin{figure}[H]
%     \centering
%     \includegraphics[width=0.6\textwidth]{assets/Provider_Addr_Item.jpg}
%     \caption{Provider overview}
%     \label{fig:Provider_addr_item}
% \end{figure}
 
% The class dedicated to the \glsplural{client} should be as simple as possible. It should provide basic interaction like
% registering, logging, deleting account, viewing product and placing order. The two last actions will stablish the communication 
% with the \glsplural{provider}.

% \begin{figure}[H]
%     \centering
%     \includegraphics[width=0.6\textwidth]{assets/client_CD.jpg}
%     \caption{Client Overview}
%     \label{fig:client_CD}
% \end{figure}

% Finally we have an order placed by a \gls{client} and processed by a \gls{provider}. Here we will rely on a third party 
% to stablish the payment procedures.

% \begin{figure}[H]
%     \centering
%     \includegraphics[width=0.6\textwidth]{assets/order_cd.jpg}
%     \caption{Order Overview}
%     \label{fig:order_cd}
% \end{figure}

\newpage
\thispagestyle{lscape}
\begin{landscape}

Zooming down the classes we presented before, we can see how they are build with its attributes, methods and
interaction with other elements.

\begin{figure}[H]
    \centering
    \includegraphics[width=1.5\textwidth]{assets/classes_CD.jpg}
    \caption{Classes Overview}
    \label{fig:class_CD}
\end{figure}

\end{landscape}

\subsection{Behavior view}

The following \gls{activity diagram} depicts the register and login procedure within the app. It should explain our
main stakeholders, \glsplural{provider} and \glsplural{client}, the starting process of the app.

\begin{figure}[H]
    \centering
    \includegraphics[width=1\textwidth]{assets/login_AC.jpg}
    \caption{Login procedures}
    \label{fig:login_register}
\end{figure}

To help understanding the above process we created the following \gls{sequence diagram} that depict the communication flow 
with the third party providers. 


\begin{figure}[H]
    \centering
    \includegraphics[width=0.7\textwidth]{assets/sequence_login_payment.jpg}
    \caption{Sequence of actions with third party applications}
    \label{fig:sequence_login_payment}
\end{figure}



